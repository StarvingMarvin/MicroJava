\chapter{Generisanje koda}

\section{Uvod u LLVM}

LLVM je...

\section{LLVM međureprezentacija}

Ceentralno mesto u arhitekturi LLVM-a\cite{aosa} zauzima međureprezentacija\footnote{eng. Intermediate Representation - IR} kojom se opisuje kod unutar sistema. LLVM IR je dizajniran da omogući analize i transformacije koda kakve se mogu očekivati u delu kompajlera koji se bavi optimizacijom. Ono što je značajno kod ove reprezentacije je da ne predstavlja neki interni implementacioni detalj, kao kod vecine kompajlera, nego je pogurana u prvi plan kao jezik jasno definisane semantike. LLVM međureprezentacija se tako sasvim ravnopravno može opisati svojom sintaksom sličnom asembleru, bajt-kodom i direktno iz programa putem C++ API-ja\footnote{eng. Application Programming Interface}.

\section{LLVM tipovi}

Primitivni: label, void, int, float, x86mmx, metadata

Izvedeni: nizovi, funkcije, pointer, structure, vector, opaque

Tipovi prvog reda: int, float, pointer, vector, struct, array, label, metadata


\section{LLVM instrukcije}

add sub...

\section{Elementi implementacije MicroJave}

